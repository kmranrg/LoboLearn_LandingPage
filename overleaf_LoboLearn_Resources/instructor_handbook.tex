\documentclass[12pt]{article}

% --- Packages ---
\usepackage[margin=1in]{geometry}
\usepackage{titlesec}
\usepackage{xcolor}
\usepackage{graphicx}
\usepackage{listings}
\usepackage{hyperref}
\usepackage{enumitem}
\hypersetup{
    colorlinks=true,
    linkcolor=Maroon,
    urlcolor=Maroon
}

% --- UNM Theme Colors ---
\definecolor{UNMRed}{HTML}{BA0C2F}
\definecolor{LightGray}{gray}{0.95}
\definecolor{DarkText}{HTML}{1E1E1E}
\definecolor{Maroon}{HTML}{800000}

\pagestyle{empty}
\titleformat{\section}{\color{UNMRed}\normalfont\Large\bfseries}{}{0em}{}

% --- Code Style ---
\lstset{
    backgroundcolor=\color{LightGray},
    basicstyle=\ttfamily\small,
    frame=single,
    breaklines=true,
    tabsize=2
}

\begin{document}

% --- Title ---
\begin{center}
    {\LARGE \bfseries \color{UNMRed}Instructor Handbook: Writing Questions}

    \vspace{1em}
    \includegraphics[width=0.20\textwidth]{logo_bold.png}
\end{center}

\vspace{-1em}
\noindent\rule{\linewidth}{1pt}

% --- Intro ---
\section*{Overview}
This short guide will help you get started with creating your own interactive question in LoboLearn. Each question typically includes two files:

\begin{itemize}
    \item \textbf{question.html} — the interface students interact with
    \item \textbf{server.py} — defines how the question is generated and graded
\end{itemize}

---

\section*{1. Frontend: question.html}
This file defines how the question is displayed to the student. Here's a simple numeric input example:

\begin{lstlisting}[language=HTML]
<pl-question-panel>
  <p>What is {{params.a}} + {{params.b}}?</p>
  <pl-number-input answers-name="ans" placeholder=""></pl-number-input>
</pl-question-panel>
\end{lstlisting}

---

\section*{2. Backend: server.py}
This file generates the random numbers used in the question and checks the student's answer.

\begin{lstlisting}[language=Python]
import random

def generate(data):
    a = random.randint(1, 10)
    b = random.randint(1, 10)
    data['params']['a'] = a
    data['params']['b'] = b
    data['correct_answers']['ans'] = a + b
\end{lstlisting}

---

\section*{3. How It Works}
When a student loads the question:
\begin{enumerate}
    \item \texttt{server.py} randomly generates values for \texttt{a} and \texttt{b}
    \item These values are passed to \texttt{question.html} using \texttt{{\{\{params.a\}\}}}
    \item When the student submits, the answer is compared to \texttt{a + b}
\end{enumerate}

---


\vfill
\begin{center}
    \textit{Need help? Contact the team at \href{mailto:lobolearn@unm.edu}{lobolearn@unm.edu}}\\[0.5em]
\end{center}

\end{document}
